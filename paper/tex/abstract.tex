
% Abstract
%% ~ 0.5 pages

Distributed training frameworks, like TensorFlow, have been proposed as a means
to reduce the training time of deep learning models by using a cluster of GPU
servers. While such speedups are  often desirable---e.g., for rapidly
evaluating new model designs---they often come with significantly higher
monetary costs due to sublinear scalability. In this paper, we investigate the
feasibility of using training clusters  composed of cheaper  \emph{transient}
GPU servers to get the benefits of distributed training without the high costs.

We conduct the first \emph{large-scale} empirical analysis, launching more than
a thousand GPU servers of various capacities, aimed at understanding the
characteristics of transient GPU servers and their impact on distributed
training performance. Our study demonstrates the potential of  transient
servers with a speedup of 7.7X with more than 62.9\% monetary savings for some
cluster configurations. We also identify a number of important challenges and
opportunities for redesigning distributed training frameworks to be
transient-aware. For example, the dynamic cost and availability characteristics
of transient servers suggest the need for frameworks to dynamically change
cluster configurations to best take advantage of current conditions. 
 
